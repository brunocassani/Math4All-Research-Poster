% unofficial poster template for Trinity College Dublin
% which is a fork of the unofficial University of Texas at Arlington Math Poster template:
% which is a fork of the unofficial University of Lethbridge Poster template: https://www.overleaf.com/latex/templates/university-of-lethbridge-unofficial-poster-template/nddfzgvqvfwf
% which is a fork of unofficial University of Alberta Poster template: 
% which is a fork of Yale template: https://www.overleaf.com/latex/templates/yale-poster-template/rjpgqfgvsjcv
% which is a fork of the UMich template https://www.overleaf.com/latex/templates/university-of-michigan-umich-poster-template/xpnqzzxwbjzc
% which is fork of the MSU template https://www.overleaf.com/latex/templates/an-unofficial-poster-template-for-michigan-state-university/wnymbgpxnnwd
% which is a fork of https://www.overleaf.com/latex/templates/an-unofficial-poster-template-for-new-york-university/krgqtqmzdqhg
% which is a fork of https://github.com/anishathalye/gemini
% also refer to https://github.com/k4rtik/uchicago-poster
% and https://www.overleaf.com/latex/templates/tcd-poster-template/gtnrnpdmqxgk

\documentclass[final]{beamer} %FIXME SELF-REFERENCE IS SO IMPORTANT. ADDRESS IT DIRECTLY, MAYBE BRING A CARD THAT SAYS "(THIS) STATEMENT IS NOT (TRUE)" AND THIS STATEMENT IS CANNOT BE PROVED

% ====================
% Packages
% ====================

\usepackage[T1]{fontenc}
\usepackage[utf8]{luainputenc}
\usepackage{lmodern}
\usepackage[size=custom, width=122,height= 91, scale=1.2]{beamerposter} %OG size=custom, width=122,height=91, scale=1.2
\usetheme{gemini}
\usecolortheme{msu}
\usepackage{graphicx}
\usepackage{booktabs}
\usepackage{tikz}
\usepackage{pgfplots}
\pgfplotsset{compat=1.14}
\usepackage{anyfontsize}

% ====================
% Lengths
% ====================

% If you have N columns, choose \sepwidth and \colwidth such that
% (N+1)*\sepwidth + N*\colwidth = \paperwidth
\newlength{\sepwidth}
\newlength{\colwidth}
\setlength{\sepwidth}{0.025\paperwidth}
\setlength{\colwidth}{0.3\paperwidth}

\newcommand{\separatorcolumn}{\begin{column}{\sepwidth}\end{column}}

% ====================
% Title
% ====================

\title{Gödel's First Incompleteness Theorem}

\author{Bruno Cassani}
% add following line if you have co-author(s)
% Coauthor One$^{2}$, Coauthor Two$^{3}$

\institute[shortinst]{\textbf{Morrissey College of Arts and Sciences}, Boston College}

% ====================
% Footer (optional)
% ====================

\footercontent{  \hfill
  \href{mailto:youremail@tcd.ie}{cassanib@bc.edu}}
% (can be left out to remove footer)

% ====================
% Logo
% ====================

% use this to include logos on the left and/or right side of the header:
% Left: institution
 \logoright{\includegraphics[height=8cm]{logos/bclogo.png}}
% Right: funding agencies and other affilations 
%\logoright{\includegraphics[height=7cm]{logos/NSF.eps}}

% ====================
% Body. Real Poster starts here
% ====================

\begin{document}



\begin{frame}[t]
\begin{columns}[t]
\separatorcolumn

\begin{column}{\colwidth}



  \begin{alertblock}{Theorem}

    In any reasonable math system, there will always be at least one true statement that cannot be proven nor disproven.

    \begin{figure}
      \centering
            \includegraphics[width=0.6\textwidth]{figures/Draw_1.png} %FIXME: a little too thick?
    \end{figure}

  \end{alertblock}
  
\begin{block}{Background}

    \begin{itemize}
      \item Before Gödel, people expected math to eventually be complete, i.e., to be able to prove everything given the right amount of axioms.
      \item In the early 20th century, set theory paradoxes like those proposed by Bertrand Russell raised questions about the consistency of math.
        
    \end{itemize}

  \end{block}

\begin{block}{Gödel Numbering}

      In order to prove the theorem, Gödel needed math to be able to talk about itself. To do so, he created his own $Encode(G)$ function to turn statements into unique numbers. To do so, he would first need to convert each mathematical symbol into a number.

    \begin{table}[h]
    \centering
    \begin{tabular}{c|c}
    \textbf{Constant Sign} & \textbf{Gödel Number} \\ \hline
    $\neg$ & 1 \\ 
    $\lor$ & 2 \\ 
    $\supset$ & 3 \\ 
    $\exists$ & 4 \\ 
    $=$ & 5 \\ 
    $\vdots$ & $\vdots$ \\ 
    \end{tabular}
    \end{table}

    %FIXME: LEAVE A LINE HERE?
    Through this system, each symbol has its own unique natural number to be used for encoding.
    
  \end{block}

\end{column}

\separatorcolumn %NEW COLUMN%

\begin{column}{\colwidth}

  \begin{block}{Encoding}

    Given a sequence of Gödel numbers $(x_1, x_2, \dots, x_n)$, the encoding is the product of the first $n$ prime numbers raised to the values in the sequence.

    $$Encode(x_1, x_2, \dots, x_n) = 2^{x_1} \times 3^{x_2} \times \ldots \times p_n^{x_n}$$

    This way, any given mathematical expression can be encoded algebraically. Note that it can also be decoded through prime factorization

    Note: $Encode(A)$ is sometimes written as $\ulcorner A \urcorner$.

  \end{block}

 \begin{block}{Provability}

Since statement A can be proved through axiom B, and $Encode(A)$ and $Encode(B)$ are unique numbers, there must be a mathematical relation between the two.

 \begin{itemize}
    
    \item We can express this relation as a function $Provability(A)$ that determines whether a statement $A$ is provable within the formal system.
    
    \item This function is essentially a binary predicate that determines if $A$ can be proved with the current axioms.
    
    \end{itemize}

 \end{block}


 \begin{block}{Proof by diagonalization}

    
    Enumerate all formulas in the formal system $F$ with exactly one free variable:
    
    \[
    \begin{array}{c|c|c|c|c}
    & n=1 & n=2 & \cdots & n = j \\
    \hline
    F_1(n) & F_1(1) & F_1(2) & \cdots & F_1(j) \\
    F_2(n) & F_2(1) & F_2(2) & \cdots & F_2(j) \\
    \vdots & \vdots & \vdots & \ddots & \vdots \\
    F_j(n) & F_j(1) & F_j(2) & \cdots & F_j(j) \\
    \end{array}
    \]


Each entry represents a formula \( F_i(n) \), where \( i \) represents the formula number and \( n \) represents the parameter.

Construct a new formula $G$, asserting the negation of provability for each formula $F_j(j)$ in the table:
\[ G \equiv \neg Provability(F_j(j)) \]

Consider the truth value of $G$. If $G$ were false, then by its own definition, each $F_j(j)$ would be provable and thus true. But the definition of $G$ implies the opposite; since math is consistent, $G$ must be true. Since $G$ cannot be consistently proven or disproven within the system, it follows that $G$ is true but unprovable within $F$.
 
 \end{block}

\end{column}


%%THIRD COLUMN
\separatorcolumn

\begin{column}{\colwidth}


  \begin{block}{References}

    \nocite{*}
    \footnotesize{\bibliographystyle{plain}\bibliography{poster}}

  \end{block}

\end{column}

\separatorcolumn
\end{columns}
\end{frame}

\end{document}
